\documentclass[]{article}
\usepackage{lmodern}
\usepackage{amssymb,amsmath}
\usepackage{ifxetex,ifluatex}
\usepackage{fixltx2e} % provides \textsubscript
\ifnum 0\ifxetex 1\fi\ifluatex 1\fi=0 % if pdftex
  \usepackage[T1]{fontenc}
  \usepackage[utf8]{inputenc}
\else % if luatex or xelatex
  \ifxetex
    \usepackage{mathspec}
  \else
    \usepackage{fontspec}
  \fi
  \defaultfontfeatures{Ligatures=TeX,Scale=MatchLowercase}
\fi
% use upquote if available, for straight quotes in verbatim environments
\IfFileExists{upquote.sty}{\usepackage{upquote}}{}
% use microtype if available
\IfFileExists{microtype.sty}{%
\usepackage{microtype}
\UseMicrotypeSet[protrusion]{basicmath} % disable protrusion for tt fonts
}{}
\usepackage[margin=1in]{geometry}
\usepackage{hyperref}
\hypersetup{unicode=true,
            pdftitle={dissertation\_writeup\_trial},
            pdfauthor={Alassane Ndour},
            pdfborder={0 0 0},
            breaklinks=true}
\urlstyle{same}  % don't use monospace font for urls
\usepackage{graphicx,grffile}
\makeatletter
\def\maxwidth{\ifdim\Gin@nat@width>\linewidth\linewidth\else\Gin@nat@width\fi}
\def\maxheight{\ifdim\Gin@nat@height>\textheight\textheight\else\Gin@nat@height\fi}
\makeatother
% Scale images if necessary, so that they will not overflow the page
% margins by default, and it is still possible to overwrite the defaults
% using explicit options in \includegraphics[width, height, ...]{}
\setkeys{Gin}{width=\maxwidth,height=\maxheight,keepaspectratio}
\IfFileExists{parskip.sty}{%
\usepackage{parskip}
}{% else
\setlength{\parindent}{0pt}
\setlength{\parskip}{6pt plus 2pt minus 1pt}
}
\setlength{\emergencystretch}{3em}  % prevent overfull lines
\providecommand{\tightlist}{%
  \setlength{\itemsep}{0pt}\setlength{\parskip}{0pt}}
\setcounter{secnumdepth}{5}
% Redefines (sub)paragraphs to behave more like sections
\ifx\paragraph\undefined\else
\let\oldparagraph\paragraph
\renewcommand{\paragraph}[1]{\oldparagraph{#1}\mbox{}}
\fi
\ifx\subparagraph\undefined\else
\let\oldsubparagraph\subparagraph
\renewcommand{\subparagraph}[1]{\oldsubparagraph{#1}\mbox{}}
\fi

%%% Use protect on footnotes to avoid problems with footnotes in titles
\let\rmarkdownfootnote\footnote%
\def\footnote{\protect\rmarkdownfootnote}

%%% Change title format to be more compact
\usepackage{titling}

% Create subtitle command for use in maketitle
\providecommand{\subtitle}[1]{
  \posttitle{
    \begin{center}\large#1\end{center}
    }
}

\setlength{\droptitle}{-2em}

  \title{dissertation\_writeup\_trial}
    \pretitle{\vspace{\droptitle}\centering\huge}
  \posttitle{\par}
    \author{Alassane Ndour}
    \preauthor{\centering\large\emph}
  \postauthor{\par}
      \predate{\centering\large\emph}
  \postdate{\par}
    \date{23/08/2019}


\begin{document}
\maketitle

\hypertarget{introduction-and-objectives}{%
\section{Introduction and
Objectives}\label{introduction-and-objectives}}

\hypertarget{background-to-the-problem}{%
\subsection{background to the problem}\label{background-to-the-problem}}

\begin{itemize}
\tightlist
\item
  Importance of growth models
\item
  Noisy environment - Systematic error
\item
  Classification using Bayes factor - e.g.~See for a dataset classified
  as linear which model seems to fit (frequentist) better which will be
  selected by BF and by how much while varying noise
\end{itemize}

\hypertarget{reasons-for-the-choice-of-project}{%
\subsection{Reasons for the choice of
project}\label{reasons-for-the-choice-of-project}}

\begin{itemize}
\tightlist
\item
  Application in biology and economics
\item
  Contribution to literature as unexplored method - combination of Bayes
  factor and Harris paper
\item
  Study of noise to signal ratio in classification
\end{itemize}

\hypertarget{identification-of-the-projects-beneficiaries}{%
\subsection{Identification of the project's
beneficiaries}\label{identification-of-the-projects-beneficiaries}}

\begin{itemize}
\tightlist
\item
  Commercial partner (probably not but mught get data from them)
\item
  Literature as an empirical analysis of Bayesian classification of
  growth using different models
\end{itemize}

\hypertarget{objectives-and-metrics}{%
\subsection{Objectives and metrics}\label{objectives-and-metrics}}

\begin{itemize}
\tightlist
\item
  A Classification framework which should include :

  \begin{itemize}
  \tightlist
  \item
    A classification between different models
  \item
    The ``certainty'' of classification - TBD how we can quantify this
  \item
    An estimation of the parameters of the model - with the
    ``certainty'' of estimation
  \item
    An identification of the systematic error
  \end{itemize}
\end{itemize}

\hypertarget{broad-methods-and-how-they-answer-goal}{%
\subsection{Broad methods and how they answer
goal}\label{broad-methods-and-how-they-answer-goal}}

\begin{itemize}
\item
  Curve fitting :

  \begin{itemize}
  \tightlist
  \item
    Fit a linear and a logistic and classify depending on the error. See
    how as you increase the variance of the error, the classification
    changes
  \end{itemize}
\item
  Bayesian approach :

  \begin{itemize}
  \item
    Estimate the distribution of the parameters (we should get the
    ``certainty'' from here) of a Bayesian linear regression, sigmoid
    function and then add the algorithm set by Harris (his calculation
    was for a sigmoid. Might have to do it for a linear regression) and
    compare the models using Bayes factor.
  \item
    Does the model that fits the most correspond to the correct
    functional form?
  \item
    See if as you change error the systematic error is caught by the
    Harris algo and how the model selection varies
  \end{itemize}
\item
  Compare the two approches : how do they compare ? In terms of
  classification error rate for instance
\item
  Furthermore, there have been interesting developments in combining
  Bayesian methods and cross validation as they are not mutually
  exclusive methods and can contribute to robust estimates. Such works
  include Bürkner et al. (2019) where the authors aim to improve upon
  leave-future-out cross-validation (LFO-CV) - an adaptation of
  leave-one-out cross-validation (LOO-CV) to timeseries - to reduce
  computation time.
\end{itemize}

\hypertarget{context-literature}{%
\section{Context (Literature)}\label{context-literature}}

\begin{itemize}
\tightlist
\item
  Ed Harris
\item
  LOO-CV
\item
  Bayesian books
\item
  Bürkner et al. (2019)
\end{itemize}

\hypertarget{data}{%
\section{Data}\label{data}}

\begin{itemize}
\tightlist
\item
  Synthetic data:

  \begin{itemize}
  \tightlist
  \item
    Two sets of data one linear and one sigmoid
  \item
    Split datasets into ratio of error to signal. 5 different variations
  \end{itemize}
\end{itemize}

\hypertarget{methods}{%
\section{Methods}\label{methods}}

\hypertarget{results}{%
\section{Results}\label{results}}

\hypertarget{discussion}{%
\section{Discussion}\label{discussion}}

\hypertarget{evaluation-reflections-and-conclusions}{%
\section{Evaluation, Reflections, and
Conclusions}\label{evaluation-reflections-and-conclusions}}

\(\sum_{i=1}^n X_i\)


\end{document}
